\documentclass[letterpaper]{article}
\usepackage[margin=1in]{geometry}
\usepackage{amsthm,amssymb,amsmath}
\title{ASON: A semantically complete superset of JSON}
\author{Casey Dahlin}
\begin{document}
\setlength{\parskip}{1em}
\setlength{\parindent}{0em}
\newtheorem{prule}{Rule}
\newcommand{\kvpair}[2]{``#1":\quad#2,\quad}
\newcommand{\kvpaire}[2]{``#1":\quad#2}

\maketitle

\section{Introduction}
JSON\footnote{RFC 4627 - http://www.ietf.org/rfc/rfc4627.txt?number=4627} has
become an increasingly popular data interchange format for a wide variety of
applications, from web services and RPC interfaces, to databases. Yet the JSON
spec has never specified a semantic for JSON values.

Meanwhile, NoSQL databases, some of which use JSON for data presentation or
more, are rising in popularity, yet there are no agreed upon semantics for
querying such a database.

ASON is a superset of JSON syntax, as well as a semantic. The acronym stands
for \textbf{A}lgebraic \textbf{S}erialized \textbf{O}bject \textbf{N}otation.
It is designed to represent not only JSON values, but families of values. It
includes a series of transforms which can allow a large, complex value or
composition of values, such as might represent an entire table in a database,
to be queried or made to conform to a schema.

It is hoped that ASON will serve not only these practical purposes, but also
open the door to a formalization for the transformation of unstructured data,
similar to what relational algebra has been for relational databases.

\section{Additional Syntax}
\begin{samepage}
We will assume a general familiarity with JSON syntax already. To JSON's
syntax, ASON adds three new invariant values, much like \(true\), \(false\),
and \(null\):

\begin{itemize}
	\item The Universe value (\(U\))
	\item The Wild value (\(*\))
	\item The Empty value (\(\emptyset\), also rendered as \(\_\))
\end{itemize}
\end{samepage}

These can appear anywhere that a JSON value could appear; as list items, values
in an object, or by themselves.

\subsection{Operators}
In addition to the new constants, ASON introduces four new operators; three
binary infix operators and one unary prefix operator.

Operators appear, as you

Operators and parenthesis behave as structural characters do in the JSON
syntax\footnote{RFC 4267, section 2}.

\begin{samepage}
The operators, in order of descending precedence, are:

\begin{itemize}
\item Complement (\(!\)) A prefix operator.
\item Append (\(.\))
\item Intersect (\(\cup\) or \(\&\))
\item Union (\(\cap\) or \(|\))
\end{itemize}
\end{samepage}

\begin{samepage}
\subsection{Universal Object Syntax}
ASON introduces a subtype of object called a "universal object." It is
represented identically to a normal object, except that after the list of
key-value pairs is one \(,\) and one \(*\). An example:
%
\begin{equation}
\{ \kvpair{num}{6} \kvpair{string}{"hello"} *\}
\end{equation}
\end{samepage}

Please note that we will typically use the term "object" to refer to either
normal or universal objects.

\section{ASON Intuitively}
The purpose of this section is to give an intuitive grasp of ASON. The next
section will endeavor to rigorously formalize it. For this reason, this section
may gloss over or ignore several issues. Those implementing the specification
should have a thorough understanding of the formal transformations in ASON.

ASON relies on a bijection between ASON values and sets. All ASON values which
reflect typical singular JSON values map to sets of cardinality 1. We state
that two ASON values are equal \em if and only if \em their mapped sets are
equal.

\subsection{Operations in ASON}
We can think of ASON values as collections of JSON values, so the ASON value
\(\{\kvpaire{foo}{8}\}\) is a set containing only the identical JSON value.

The purpose of the \(\cup\) and \(\cap\) operators is thus apparent: they
combine and manipulate these singleton collections to form collections
containing multiple values. Their specific function is upon the underlying
sets, and analogous to the equivalent set theory operations: the union of two
ASON values is an ASON value whose underlying set is the union of each of the
values' underlying set. So the ASON value \(6\) is a collection containing
\(6\), and the ASON value \(6\cup7\) is a collection containing \(6\) and
\(7\).

\subsection{The Universe, Wild, and Empty Values}
The additional values we have added over JSON can all be explained in these
terms: \(\emptyset\) is a value underlied by the empty set, and represents a
collection containing nothing. \(U\) is the ASON universe, and is a collection
of all possible ASON values. \(*\) is similar to \(U\), except it does not
contain the value \(null\).

\begin{samepage}
\subsection{Universal Objects}
Universal objects are used to refer to collections of objects which all have
certain keys in common. For example, the collection of all objects which have a
key \(``foo"\) with value \(6\) would be:
%
\begin{equation}
\{\kvpair{foo}{6} *\}
\end{equation}
\end{samepage}

\subsection{Complementation}
The \(!\) operator also specifies a union of several values. Specifically it
specifies a union of all values not equal to the value it is applied to.

It is intuitive to say the collection of all values which are not \(6\) is
\(!6\). It is possibly more subtle to point out \(U = !\emptyset\), or \(* =
!null\), but these equalities should hold given our definition.

\subsection{Append}
The append operator only makes sense when it is between two lists, or two
objects. It can be distributed over collections of objects, however.

For lists, the behavior of append is intuitive from its name. It combines two
lists into a single list, with the elements of the first list followed by the
elements of the second. This makes append non-commutative, and also the only
operator whose behavior can't be modeled as elementary set operations.

For objects, the behavior is similar. An object is yielded with the combined
set of key-value pairs from either operand. If either operand was a universal
object, the result is as well. The only unusual case is when both operands
specify the same key. In cases where one operand specifies a key \(``foo"\) to
the value \(a\), and the other operand specifies the key \(``foo"\) to the
value \(b\), the result will have the key \(``foo"\) with the value \(a\cap b\).

We're papering over some subtlety here, but we'll allow the unusual cases to be
worked out in the formalization.

\section{Formalization}
\begin{samepage}
We will allow for numbers in JSON/ASON form to be transformed in ways that
mathematics implies yield identical results, for example \(6.0 = 6\) and \(+3 =
3\). We will also assume that characters in strings may be replaced with their
equivalent unicode escape sequences as specified by JSON, and vice versa.
Given this and the transformations on other values we will outline below, we
will define equality thus:
%
\begin{prule}
Two ASON values are equal \em if and only if \em they can each be transformed
into an identical representation.
\end{prule}
\end{samepage}

\begin{samepage}
Reflexively, this specification shall describe a series of transformations on
ASON objects which do not alter their semantic value. We will also state that:
%
\begin{prule}
\label{sec:commute}
If a value \(A\) can be transformed into a value \(B\), then the value \(B\)
may be transformed into the value \(A\).
\end{prule}
\end{samepage}

And lastly:
%
\begin{prule}
All operators are associative.
\end{prule}

\subsection{Basic Transforms}
These are intuitive rules understood by most existing JSON users and
implementations. We will formalize them here.

\begin{prule}
The order of two key-value pairs in an object may be exchanged.
\end{prule}

\begin{prule}
Values within objects or lists may be transformed.
\end{prule}

\subsection{Key Expansion}
\begin{prule}
\begin{equation}
\begin{split}
& \{\kvpaire{foo}{6} \}\\
= \quad& \{\kvpair{foo}{6} \kvpaire{bar}{null}\}
\end{split}
\end{equation}

A non-universal object may have a key-value pair with any key and value
\(null\) inserted in to it.
\end{prule}

\begin{prule}
\begin{equation}
\begin{split}
& \{\kvpair{foo}{6} *\}\\
= \quad& \{\kvpair{foo}{6} \kvpair{bar}{U} *\}
\end{split}
\end{equation}

A universal object may have a key-value pair with any key and value \(U\)
inserted in to it.
\end{prule}

Key expansion dramatically simplifies the definition of other operations.
Performing key expansion is often a crucial step in computing reduced ASON
values. Do however note that key expansion implies that there is no difference
between a key being set to \(null\) and a key not being present, at least
within non-universal objects. This is a logical, but not necessarily expected
semantic for JSON's \(null\) value.

Reversal of this transformation via rule \ref{sec:commute} is typically
referred to as \em key contraction \em .

\subsection{Obliteration}
\begin{prule}
\begin{equation}
\begin{split}
{[}7,\quad8,\quad9,\quad\emptyset,\quad10] & = \emptyset \\
\{\kvpair{foo}{6}\kvpaire{bar}{\emptyset}\} & = \emptyset
\end{split}
\end{equation}
Any object or list containing a value which is equal to \(\emptyset\) may be
replaced with \(\emptyset\)
\end{prule}

Obliteration is somewhat counterintuitive, but is required to simplify other
operations. Without this transformation it might be provable that the
underlying set of a non-universal object has a cardinality greater than 1.
Consider our coming definition of intersection.

\subsection{Append Reduction}

\begin{prule}
\begin{equation}
\begin{split}
& \Bigg( \{\kvpaire{foo}{6}\} \cup \{\kvpaire{bar}{7}\} \Bigg) \quad . \quad
\{\kvpaire{baz}{8}\} \\
= \quad& \{\kvpaire{foo}{6}\} . \{\kvpaire{baz}{8}\} \quad \cup \quad \{\kvpaire{bar}{7}\}
 . \{\kvpaire{baz}{8}\}
\end{split}
\end{equation}

The append operator may be distributed across the union operator.
\end{prule}

This is a logical rule, but its implications are significant. Because of the
obliteration rule, append can behave much like relational algebra's inner join,
crossing two unioned collections of values.

\begin{prule}
\begin{equation}
\begin{array}{rclcl}
6 \quad&.&\quad 7 & = & \emptyset \\
6 \quad&.&\quad 6 & = & \emptyset \\
\{\kvpaire{foo}{"bar"}\} \quad&.&\quad 7 & = & \emptyset \\
{[}4,\quad5,\quad6] \quad&.&\quad 7 & = & \emptyset \\
{[}4,\quad5,\quad6] \quad&.&\quad \{\kvpaire{foo}{"bar"}\} & = & \emptyset \\
\end{array}
\end{equation}

If either operand of an append is \(true\), \(false\), \(null\), \(\emptyset\),
a string, or a number, or if one operand is a list and the other an object, the
append may be reduced to \(\emptyset\)
\end{prule}

Less formally, things which do not make sense to append to oneanother become
\(\emptyset\) when appended. Because, \(\emptyset\) tends to reduce out
of ASON values due to the obliteration rule and properties of unions which we
will see later, this is a rather clean way to handle these cases, particularly
as opposed to simply declaring them badly formed.

\begin{prule}
\begin{equation}
\begin{split}
& {[}4,\quad5,\quad6] \quad.\quad {[}7,\quad8,\quad9] \\
=\quad & {[}4,\quad5,\quad6,\quad7,\quad8,\quad9]
\end{split}
\end{equation}

An append of two lists can be reduced to a single list, with the items within
the first list followed by the items within the second list.
\end{prule}

This more or less matches our intuitive understanding of "appending" for lists
in other contexts.

\begin{prule}
\begin{equation}
\begin{array}{rlllll}
& \{\kvpair{foo}{a}&\kvpair{bar}{b}&\kvpaire{baz}{c}&&\}\quad.\\
& \{&\kvpair{bar}{d}&\kvpair{baz}{null}&\kvpaire{bam}{e}&\} \\
\\
=&\{\kvpair{foo}{a}&\kvpair{bar}{b}&\kvpair{baz}{c}&\kvpaire{bam}{null}&\}\quad.\\
 &\{\kvpair{foo}{null}&\kvpair{bar}{d}&\kvpair{baz}{null}&\kvpaire{bam}{e}&\} \\
\\
=&\Bigg\{\kvpair{foo}{a}&\kvpair{bar}{b\cap
d}&\kvpair{baz}{c}&\kvpaire{bam}{e}&\Bigg\}\\
\end{array}
\end{equation}

An append of two or more objects, universal or otherwise, can be replaced with
a single object, where that object is universal if and only if either of the
two operands were universal.
\begin{enumerate}
\item Use key expansion so both original objects have the same set of keys. The
result object will also have that set of keys
\item If either operand has \(null\) as the value for a given key, the result
object will have the value from the other operand. If both objects have
\(null\) the result object will have \(null\).
\item If neither operand has \(null\) for a given key, then the result object
will have the intersection of the two values in the operands for that key.
\end{enumerate}
\end{prule}

Consider for a moment the implication that this has for objects with duplicate
keys. JSON does not specify a semantic, and thus does not expressly prohibit
duplicate keys, although this is incompatible with most programmers' intuitive
understanding of what a JSON object represents. ASON offers a solution, where:

\begin{equation}
\{\kvpair{foo}{a}\kvpaire{foo}{b}\}
\end{equation}

Is interpreted as:

\begin{equation}
\{\kvpaire{foo}{a}\}.\{\kvpaire{foo}{b}\}
\end{equation}

And thus reduces, by the above transform, to:

\begin{equation}
\{\kvpaire{foo}{a\cap b}\}
\end{equation}

\subsection{Intersection Reduction}

\begin{prule}
\begin{equation}
\begin{split}
& \Bigg( \{\kvpaire{foo}{6}\} \cup \{\kvpaire{bar}{7}\} \Bigg) \quad \cap \quad
\{\kvpaire{baz}{8}\} \\
= \quad& \{\kvpaire{foo}{6}\} \cap \{\kvpaire{baz}{8}\} \quad \cup \quad \{\kvpaire{bar}{7}\}
 \cap \{\kvpaire{baz}{8}\}
\end{split}
\end{equation}

The intersect operator may be distributed over the union operator.
\end{prule}

This isn't at all surprising, and if we've begun to intuit the analogy between
ASON union and intersection and set theory's implementations of the same, it
becomes obvious.

\begin{prule}
\label{sec:interelim}
\begin{equation}
\begin{array}{rclcl}
6 \quad&\cap&\quad 7 & = & \emptyset \\
\{\kvpaire{foo}{"bar"}\} \quad&\cap&\quad 7 & = & \emptyset \\
{[}4,\quad5,\quad6] \quad&\cap&\quad 7 & = & \emptyset \\
{[}4,\quad5,\quad6] \quad&\cap&\quad {[}7,\quad8] & = & \emptyset \\
{[}4,\quad5,\quad6] \quad&\cap&\quad \{\kvpaire{foo}{"bar"}\} & = & \emptyset \\
\end{array}
\end{equation}

The intersection of any two values which are not equal, and of which neither
can be expressed as a union of two or more non-equal, non-empty values, is
\(\emptyset\).
\end{prule}

This is a necessarily complex way of specifying what is ultimately a simple
concept: intersecting unequal values which are not collections of multiple
values yields an empty value. It becomes muddled when we attempt to express it
as purely transformation over syntax, as this specification aims to do.

\begin{prule}
\label{sec:interident}
\begin{equation}
\begin{array}{rclcl}
6 \quad&\cap&\quad 6 & = & 6 \\
\{\kvpaire{foo}{"bar"}\} \quad&\cap&\quad \{\kvpaire{foo}{"bar"}\}  & = & \{\kvpaire{foo}{"bar"}\} \\
{[}4,\quad5,\quad6] \quad&\cap&\quad {[}4,\quad5,\quad6] & = & {[}4,\quad5,\quad6] \\
\end{array}
\end{equation}

The intersection of any two values which are equal is equal to those values.
\end{prule}

The counterpart to the previous rule is much more cleanly specified. A value
intersected with itself is itself.

\begin{prule}
\begin{equation}
{[}a,\quad b,\quad c] \quad\cap\quad {[}d,\quad e,\quad f]  =  {[}a\cap d,\quad
b\cap e,\quad c\cap f] \\
\end{equation}

The intersection of any two lists of the same length can be reduced to a list
which is also that length and where each element is the intersection of the
corresponding two elements in the operand lists.
\end{prule}

This can almost be inferred from the previous rules and some of the properties
of unions discussed later. In fact you can prove rules \ref{sec:interident} and
\ref{sec:interelim} for the specific case of lists of the same length with this
rule. Nonetheless this rule does clarify the details of intersection.

\begin{prule}
\begin{equation}
\begin{array}{rlllll}
& \{\kvpair{foo}{a}&\kvpair{bar}{b}&\kvpaire{baz}{c}&&\}\quad\cap\\
& \{&\kvpair{bar}{d}&\kvpair{baz}{null}&\kvpaire{bam}{e}&\} \\
\\
=&\{\kvpair{foo}{a}&\kvpair{bar}{b}&\kvpair{baz}{c}&\kvpaire{bam}{null}&\}\quad\cap\\
 &\{\kvpair{foo}{null}&\kvpair{bar}{d}&\kvpair{baz}{null}&\kvpaire{bam}{e}&\} \\
\\
=&\Bigg\{\kvpair{foo}{a\cap null}&\kvpair{bar}{b\cap
d}&\kvpair{baz}{c\cap null}&\kvpaire{bam}{e\cap null}&\Bigg\}\\
\end{array}
\end{equation}

An intersection of two or more objects, universal or otherwise, can be replaced
with a single object, where that object is universal if and only if both of
the two operands were universal.
\begin{enumerate}
\item Use key expansion so both original objects have the same set of keys. The
result object will also have that set of keys
\item For each key, the result object will have the intersection of the two
values for that key in the operands.
\end{enumerate}
\end{prule}

This complements the previous rule, and rounds out our definition of
intersection. Again, it nicely parallels rule \ref{sec:interelim}, but does
resolve real ambiguity in the system.

\subsection{Union Reduction}
\begin{prule}
\begin{equation}
a\cup\emptyset = a
\end{equation}

Any value unioned with \(\emptyset\) can be reduced to that value singly.
\end{prule}

This more or less defines \(\emptyset\). Combined with obliteration, it means
that \(\emptyset\) should almost always be able to be removed from an ASON
value by transformation.

\begin{prule}
\begin{equation}
a\cup a = a
\end{equation}

Any value unioned with itself can be reduced to that value singly.
\end{prule}

Interestingly, this parallels intersection.

\begin{prule}
\begin{equation}
\begin{array}{rcl}

{[}a,\quad b,\quad c] \cup {[}a,\quad b,\quad d] &=& {[}a,\quad b,\quad c \cup d] \\
\\
\{\kvpair{foo}{a}\kvpaire{bar}{b}\} \cup \{\kvpair{foo}{a}\kvpaire{bar}{c}\}
&=& \{\kvpair{foo}{a}\kvpaire{bar}{b \cup c}\}

\end{array}
\end{equation}
Any union of lists or objects where the operands  differ in only one value may
be reduced to a version of that list or object where the differing value is a
union of the values within the two operands.
\end{prule}

This rule is key, but it is more often used in reverse. Unions can be
propagated upward such that there is only one top-level sequence of unions by
repeatedly exchanging values containing unions for unions of values.

\subsection{Complementation}
\begin{prule}
Given a value \(a\), if one takes the set of all values \(x\) such that \(a
\cap x = \emptyset\) and joins them with the union operator, one arrives at a
form equivalent to \(!a\)
\end{prule}

This rule is difficult, as it describes a transformation to and from a form
which would be uncoutably infinite in length. Nonetheless it is the simplest
definition of complementation.

\subsection{Constants}
\begin{equation}
\begin{split}
U = &!\emptyset \\
* = &!null
\end{split}
\end{equation}

With complementation defined, the two constant values are easily expressed. In
fact they are revealed as syntactic sugar.

\end{document}
